\part{cloud-init}

%%%%%%%%%%%%%%%%%%%%%%%%%%%%%%%%%%%%%%%%%%%%%%%%%%%%%%%%%%%%%%%%%%%%%%%%%%%

\begin{frame}
  \frametitle{cloud-init}
  \framesubtitle{What is cloud-init?}
    \begin{itemize}
      \item Cloud-init abstracts these mechanisms and defines a format for the data
      \item It can:
      \begin{itemize}
        \item configure network, users, ssh keys, filesystems, \dots
        \item install packages,
        \item execute arbitrary commands,
        \item execute user provied scripts,
        \item configure VM with puppet or chef,
        \item \dots
      \end{itemize}
      \item It also provides mechanisms for extension.
    \end{itemize}
\end{frame}

%%%%%%%%%%%%%%%%%%%%%%%%%%%%%%%%%%%%%%%%%%%%%%%%%%%%%%%%%%%%%%%%%%%%%%%%%%%

\begin{frame}
  \frametitle{cloud-init}
  \framesubtitle{Supported platforms}

  \begin{itemize}
    \item Supports:
    \begin{itemize}
       \item EC2 metadata server (used by OpenStack)
       \item NoCloud data format as used by Synnefo.
       \item OpenNebula context.sh format included in v0.7.3
        \begin{itemize}
          \item Since v0.7.5 also supports base64 encoded data as provided in FedCloud.
        \end{itemize}
    \end{itemize}
    \item Packages available for most Linux distributions: ubuntu/debian, SL5/SL6 (in EPEL), SUSE, \dots
    \end{itemize}
\end{frame}

%%%%%%%%%%%%%%%%%%%%%%%%%%%%%%%%%%%%%%%%%%%%%%%%%%%%%%%%%%%%%%%%%%%%%%%%%%%

\begin{frame}[fragile]
  \frametitle{Data available at the VM}
  \framesubtitle{user data vs meta data}

  \begin{columns}
    \column{0.5\textwidth}
    \begin{block}{meta data}
    \begin{itemize}
      \item Basic information on the VM 
      \item VM Identifier 
      \item Hostname, IP
      \item User Public Keys
    \end{itemize}
    \end{block}

    \column{0.5\textwidth}
    \begin{block}{user data}
      User data is treated as opaque data: what you give is what
      you get back. It is up to the instance (cloud-init) to be able to
      interpret it.  
    \end{block}
  \end{columns}

  \hfil\\

  \begin{Sbox}
  \Fontsmaller
  \begin{minipage}{\linewidth-2\fboxsep-2\fboxrule-4pt}
  \color{white}
  \begin{verbatim}
~$ occi -e $ENDPOINT \
        -n x509 -X -x $X509_USER_PROXY \
        -a create -r compute  \
        -M $OS_TPL -M $RESOURCE_TPL \
        -t occi.core.title="MyTutorialContextVM_$(date +%s)" \
        -T public_key="<public key>" \
        -T user_data="<user data>"
  \end{verbatim}
  \end{minipage}
  \end{Sbox}
  \fcolorbox{black}{black}{\TheSbox}
\end{frame}

%%%%%%%%%%%%%%%%%%%%%%%%%%%%%%%%%%%%%%%%%%%%%%%%%%%%%%%%%%%%%%%%%%%%%%%%%%%

\begin{frame}[fragile]
  \frametitle{cloud-init}
  \framesubtitle{Handling the meta-data}

  Create key (if you don't have one):
  \begin{Sbox}
  \Fontsmaller
  \begin{minipage}{\linewidth-2\fboxsep-2\fboxrule-4pt}
  \color{white}
  \begin{verbatim}
~$ ssh-keygen -f test.key
  \end{verbatim}
  \end{minipage}
  \end{Sbox}
  \fcolorbox{black}{black}{\TheSbox}

  \hfill\\

  Launch VM with \rocci-cli:
  \begin{Sbox}
  \Fontsmaller
  \begin{minipage}{\linewidth-2\fboxsep-2\fboxrule-4pt}
  \color{white}
  \begin{verbatim}
~$ occi -e $ENDPOINT \
        -n x509 -X -x $X509_USER_PROXY \
        -a create -r compute  \
        -M $OS_TPL -M $RESOURCE_TPL \
        -t occi.core.title="MyTutorialContextVM_$(date +%s)" \
        -T public_key="file://$PWD/test.key.pub" 
  \end{verbatim}
  \end{minipage}
  \end{Sbox}
  \fcolorbox{black}{black}{\TheSbox}

  \hfill\\

  Check results:
  \begin{Sbox}
  \Fontsmaller
  \begin{minipage}{\linewidth-2\fboxsep-2\fboxrule-4pt}
  \color{white}
  \begin{verbatim}
~$ ssh -i test.key ubuntu@193.144.35.84
ubuntu@ip-193-144-35-84:~$ 
  \end{verbatim}
  \end{minipage}
  \end{Sbox}
  \fcolorbox{black}{black}{\TheSbox}

\end{frame}

%%%%%%%%%%%%%%%%%%%%%%%%%%%%%%%%%%%%%%%%%%%%%%%%%%%%%%%%%%%%%%%%%%%%%%%%%%%

\begin{frame}
  \frametitle{cloud-init}
  \framesubtitle{user-data}

    \begin{block}{cloud-init inspects the user-data}
    Some of the supported formats:
    \begin{itemize}
      \item user script: execute it.  (begins with \texttt{\#!})
      \item Cloud Config Data: cloud-config is the simplest way to accomplish
      some things via user-data. Using cloud-config syntax, the user can
      specify certain things in a human friendly format. (begins with 
      \texttt{\#cloud-config})
      \item include file: contains a list
      of urls, one per line. Each of the URLs will be read, and their
      content will be passed through this same set of rules. (begins with 
      \texttt{\#include})
      \item gzipped content $\rightarrow$ uncompress and use as it were
         not compressed.
      %\item If it can be executed, it is executed 
      %\item VM Identifier 
      %\item Hostname, IP
      %\item User Public Keys
    \end{itemize}
    \end{block}
\end{frame}

%%%%%%%%%%%%%%%%%%%%%%%%%%%%%%%%%%%%%%%%%%%%%%%%%%%%%%%%%%%%%%%%%%%%%%%%%%%

\begin{frame}[fragile]
  \frametitle{cloud-init}
  \framesubtitle{execute script}

  Create simple script:
  \begin{Sbox}
  \Fontsmaller
  \begin{minipage}{\linewidth-2\fboxsep-2\fboxrule-4pt}
  \color{white}
  \begin{verbatim}
#!/bin/sh
echo "Hello World." > /root/context.txt
echo "The time is now $(date -R)!" >> /root/context.txt
  \end{verbatim}
  \end{minipage}
  \end{Sbox}
  \fcolorbox{black}{black}{\TheSbox}

  \hfill\\

  Launch VM with \rocci-cli:
  \begin{Sbox}
  \Fontsmaller
  \begin{minipage}{\linewidth-2\fboxsep-2\fboxrule-4pt}
  \color{white}
  \begin{verbatim}
~$ occi -e $ENDPOINT -n x509 -X -x $X509_USER_PROXY \
        -a create -r compute  \
        [...]
        -T public_key="file://$PWD/test.key.pub" \
        -T user_data="file://$PWD/script.sh"
  \end{verbatim}
  \end{minipage}
  \end{Sbox}
  \fcolorbox{black}{black}{\TheSbox}

  \hfill\\

  Check results:
  \begin{Sbox}
  \Fontsmaller
  \begin{minipage}{\linewidth-2\fboxsep-2\fboxrule-4pt}
  \color{white}
  \begin{verbatim}
~$ ssh -i test.key ubuntu@155.210.71.129 "sudo cat /root/context.txt"
Hello World
The time is now Sat, 13 Dec 2014 16:01:29 +0100
  \end{verbatim}
  \end{minipage}
  \end{Sbox}
  \fcolorbox{black}{black}{\TheSbox}

\end{frame}
