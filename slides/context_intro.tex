\part{ContextIntro}

%%%%%%%%%%%%%%%%%%%%%%%%%%%%%%%%%%%%%%%%%%%%%%%%%%%%%%%%%%%%%%%%%%%%%%%%%%%
\begin{frame}
  \frametitle{Contextualization}
  \framesubtitle{What -- Why?}

  \begin{block}{What?}
    Contextualization is the process of installing, configuring and
    preparing software upon boot time on a pre-defined virtual machine
    image (e.g. hostname, IP address, ssh keys, \dots)
  \end{block}

  \begin{block}{Why?}
    \begin{itemize}
      \item Configuration not known until instantiation (e.g. data location).
      \item Private Information (e.g. host certs)
      \item Software that changes frequently or under development.
      \item Not practical to create a new VM image for every possible configuration.
    \end{itemize}
  \end{block}
\end{frame}

%%%%%%%%%%%%%%%%%%%%%%%%%%%%%%%%%%%%%%%%%%%%%%%%%%%%%%%%%%%%%%%%%%%%%%%%%%%

\begin{frame}
  \frametitle{Contextualization}
  \framesubtitle{How?}

  \begin{block}{How?}
    \begin{itemize}
      \item Contextualization requires passing some data to the VMs on
            instantiation
      \item Standard OCCI API lacks such feature
      \item New mixins proposed and implemented in \rocci-server, OCCI-OS and Synnefo (i.e. all current RPs of EGI's Fedcloud!)
    \end{itemize}
  \end{block}

\end{frame}

%%%%%%%%%%%%%%%%%%%%%%%%%%%%%%%%%%%%%%%%%%%%%%%%%%%%%%%%%%%%%%%%%%%%%%%%%%%

\begin{frame}[fragile]
  \frametitle{Contextualization}
  \framesubtitle{OCCI extensions}

  User data:
  \begin{Sbox}
  \Fonttiny
  \begin{minipage}{\linewidth-2\fboxsep-2\fboxrule-4pt}
  \color{white}
  \begin{verbatim}
Category: user_data;
          scheme="http://schemas.openstack.org/compute/instance#";
          class="mixin”

X-OCCI-Attribute: org.openstack.compute.user_data="<base64 encoded data>"
  \end{verbatim}
  \end{minipage}
  \end{Sbox}
  \fcolorbox{black}{black}{\TheSbox}

  \hfill\\

  Public SSH keys:
  \begin{Sbox}
  \Fonttiny
  \begin{minipage}{\linewidth-2\fboxsep-2\fboxrule-4pt}
  \color{white}
  \begin{verbatim}
Category: public_key;
          scheme="http://schemas.openstack.org/instance/credentials#";
          class="mixin"

X-OCCI-Attribute: org.openstack.credentials.publickey.name="<key name>"
X-OCCI-Attribute: org.openstack.credentials.publickey.data="<the public key>"
  \end{verbatim}
  \end{minipage}
  \end{Sbox}
  \fcolorbox{black}{black}{\TheSbox}

\end{frame}

%%%%%%%%%%%%%%%%%%%%%%%%%%%%%%%%%%%%%%%%%%%%%%%%%%%%%%%%%%%%%%%%%%%%%%%%%%%

\begin{frame}[fragile]
  \frametitle{Contextualization}
  \framesubtitle{\rocci client}

  Use \texttt{-T} option with \rocci-cli:

  \begin{Sbox}
  \Fontsmaller
  \begin{minipage}{\linewidth-2\fboxsep-2\fboxrule-4pt}
  \color{white}
  \begin{verbatim}
~$ occi -e $ENDPOINT \
        -n x509 -X -x $X509_USER_PROXY \
        -a create -r compute  \
        -M $OS_TPL -M $RESOURCE_TPL \
        -t occi.core.title="MyTutorialContextVM_$(date +%s)" \
        -T public_key="<public key>" \
        -T user_data="<user data>"
  \end{verbatim}
  \end{minipage}
  \end{Sbox}
  \fcolorbox{black}{black}{\TheSbox}
\end{frame}

%%%%%%%%%%%%%%%%%%%%%%%%%%%%%%%%%%%%%%%%%%%%%%%%%%%%%%%%%%%%%%%%%%%%%%%%%%%

\begin{frame}
  \frametitle{Contextualization}
  \framesubtitle{Using the context data}

    \begin{itemize}
        \item OCCI extensions specifies how to pass data to the VM
        \item BUT not how the data will be available!
        \item Each RP has different mechanisms to provide the data:
        \begin{itemize}
            \item metadata server at a fixed location
            \item iso filesystem
            \item file injection
            \item \dots
        \end{itemize}
        \item Need a way to abstract this mess!
    \end{itemize}
\end{frame}

%%%%%%%%%%%%%%%%%%%%%%%%%%%%%%%%%%%%%%%%%%%%%%%%%%%%%%%%%%%%%%%%%%%%%%%%%%%


